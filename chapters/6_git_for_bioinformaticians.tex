\documentclass[../main/git_course_main.tex]{subfiles}
\begin{document}
	
	\setcounter{chapter}{5}
	\chapter{Version control for bioinformaticians}
	
	% Tagging commits and .gitignore
	
	\section{Overview}
	
	Here, we will discuss Git-concepts which are especially useful in the
	context of bioinformatics.
	
	\begin{itemize}
		\item Excluding files residing in the file tree from the repository
		\item Version-tracking by tagging commits
	\end{itemize}
	
	\section{Keeping the right files in version control}
	
	You sometimes encounter a situation where you want to keep files within the file-tree of a repository without having the repository track them. For example:
	
	\begin{enumerate}
		\item Either your editor or your code generates temporary files when in use.
		\item Configuration files which are linked to one specific computer.
		\item You want to keep data (for example - test datasets) in the directory of a script.
	\end{enumerate}
	
	If the files are computer generated and automatically updating, or linked to some setup of your particular computer, you should probably exclude them from the repository. For example: This document is written in LaTeX and is version controlled in Git, but temporary files generated by the text editor (for example when generating the PDFs) are excluded from the repository.
	
	It is generally not a good idea to track bioinformatic data in a repository. It is definitely not a good idea to track or try to push 10GB of sequence data to GitHub. Both GitHub and Bitbucket will both reject pushes containing too large files. It can sometimes
	be useful to keep very small test-datasets in the repository, but you should think carefully before committing non-source code files to the repository.
	
	\subsection{.gitignore}
	
	To exclude particular folders or file-patterns from the Git repository, special files named '.gitignore' are used.
	To use those, simply create a file named '.gitignore' in the directory you want to control, and it will affect all its sub-directories.
	
	To exclude a specific file, simply put its path into a \verb$.gitignore$ file.
	Relative paths will start from the directory of the \verb$.gitignore$ file.
	
	\begin{codebox}
		\begin{lstlisting}
			$ echo "Subfolder/exact_file.log" >> .gitignore
		\end{lstlisting}
	\end{codebox}
	
	To exclude all files in a directory, add the path to the directory.
	For example, if we want to exclude all files in the "Subfolder" directory, then we would use the following:
	
	\begin{codebox}
		\begin{lstlisting}
			$ echo "Subfolder/" >> .gitignore
		\end{lstlisting}
	\end{codebox}
	
	There are also some patterns we can use to match particular suffixes or directory names, excluding files matching a particular pattern. An asterisk (*) can be used to match part of a file name. Two asterisks can be used to match any number of directories (**).
	
	If we for example want to exclude all files with the suffix .log, we would do the following:
	
	\begin{codebox}
		\begin{lstlisting}
			$ echo "*.log" >> .gitignore
		\end{lstlisting}
	\end{codebox}
	
	If we want to exclude all directories named "output" with all their sub-files, we would do the following:
	
	\begin{codebox}
		\begin{lstlisting}
			$ echo "**output/" >> .gitignore
		\end{lstlisting}
	\end{codebox}
	
	\subsection{Removing a file from the repository, while keeping it in the file tree}
	
	The changes of the \verb$.gitignore$-file is seen in the \verb$git status$
	output as soon as the file is being tracked by the repository. This will not affect already committed files - It only ignores files which aren't being tracked. Files that already have been committed need to be removed from the repository before being ignored.
	
	To remove files from the repository without removing them from the file tree, we can use the \verb$git rm --cached$ command (add the \verb$-r$ flag to traverse down into directories).
	
	If we for example want to remove all log-files that currently are being tracked, we would do the following:
	
	\begin{codebox}
		\begin{lstlisting}
			$ git rm -r --cached *.log
			rm 'file1.log'
			rm 'file2.log'
		\end{lstlisting}
	\end{codebox}
	
	If we want to remove the \verb$subdir/$ directory from the repository, we would do the following:
	
	\begin{codebox}
		\begin{lstlisting}
			git rm -r --cached subdir/
			rm 'subdir/README.txt'
			rm 'subdir/hello_world.cpp'
		\end{lstlisting}
	\end{codebox}
	
	Note that this doesn't remove it from neither the file-tree, nor Git's \verb$history$. It simply produces a commit where the files are removed from being tracked. They will then show up when running \verb$git status$ until you have entered them into your \verb$.gitignore$ file.
	
	Further examples of usages for the \verb$.gitignore$ file is found at: \\
	
	\url{https://git-scm.com/docs/gitignore}
	
	\section{How to keep track of versions}
	
	When doing bioinformatics, it is imperative to at all times be able to reproduce all the steps
	performed in your analysis. You should at all times be able to replicate your analysis exactly. Science requires the ability for you and others to replicate your findings.
	This means that you need to be able to trace the exact commands that you have been using in your analysis.
	Furthermore, it should be possible to use the \textit{exact same versions} of all the software you have been using.
	
	Git is great help for helping keeping track of software versions. If you keep your script in version control at all times, you will know that the version
	of the code that you used is there, and you could easily keep track of which exact version that you have been using.
	
	\subsection{Tagging commits}
	
	It is often useful to tag particular
	commits linked to the versions. This makes it easy to at a later point revisit important stages of the script,
	and you are able to refer to the versions used in the documentation of your analysis.
	
	You can use the \verb$git tag$ command to tag your current commit. Providing the tag name using the \verb$--annotate$ (short: \verb$-a$) flag and the description using the \verb$--message$ (short: \verb$-m$) flag.
	
	\begin{codebox}
		\begin{lstlisting}
			$ git tag -a v1.0 -m "This version was a major revision of.."
		\end{lstlisting}
	\end{codebox}
	
	You can list all available tags with the \verb$git tag$ command.
	
	\begin{codebox}
		\begin{lstlisting}
			$ git tag
			v1.0
		\end{lstlisting}
	\end{codebox}
	
	To get further information about the tag, you can run the \verb$git show$ command:
	
	\begin{codebox}
		\begin{lstlisting}
			$ git show v1.0
			tag v1.0
			Tagger: Jakob Willforss <jakob.willforss@hotmail.com>
			Date:   Thu May 12 13:38:55 2016 +0200
			
			My description!
			
			commit 3ab37147b530336b09913a0afdd1fa910c041149
			Merge: cf8a9d5 984f0ac
			Author: Jakob <jakob.willforss@hotmail.com>
			Date:   Wed May 11 14:11:07 2016 +0200
			
			Merge branch 'main' of https://github.com/Jakob37/GitCourse
		\end{lstlisting}
	\end{codebox}
	
	%%%% GIT TAG %%%%%
	\begin{figure}[h!]
		\begin{bluebox}
			Command: \verb$git tag -a -m$ \\
			Command: \verb$git tag$ \\
			
			Create a new tag for the current commit with target annotation and message.
			Use the command without any argument to list all current tags.
		\end{bluebox}
		\label{command:tag}
		\caption{git tag - Tag a commit for easy access}
	\end{figure}
	%%%%%%%%%%%%%%%%%%%%%%
	
	%%%% GIT SHOW %%%%%
	\begin{figure}[h!]
		\begin{bluebox}
			Command: \verb$git show <tag>$ \\
			
			Show more information about target object (for example a tag).
		\end{bluebox}
		\label{command:tag}
		\caption{git show - Show more information about objects like tags}
	\end{figure}
	%%%%%%%%%%%%%%%%%%%%%%
	
	\subsection{Checking out tags}
	
	There are two ways to check out the tagged version of the file tree:
	
	\begin{itemize}
		\item Use the \verb$git show$ command to get the commit ID, and run \verb$git checkout$ for that
		\item Create and check out a new branch/head based on the tag
	\end{itemize}
	
	Example of checking out tag by creating a new branch.
	
	\begin{codebox}
		\begin{lstlisting}
			$ git checkout -b version1.0 v1.0
		\end{lstlisting}
	\end{codebox}
	
	\subsection{Pushing tags to remote}
	
	The tags you add are per default only present on the local branch. If you want them to be
	pushed to the remote and made available for your collaborators (or yourself on GitHub), you need to specify that you want to push them too.
	
	You can push them one and one by using their names. To push many tags at once, the flag \verb$--follow-tags$ is recommended.
	
	\begin{codebox}
		\begin{lstlisting}
			$ git push origin v1.0
			Username for 'https://github.com': Jakob37
			Password for 'https://Jakob37@github.com': 
			Counting objects: 1, done.
			Writing objects: 100% (1/1), 173 bytes | 0 bytes/s, done.
			Total 1 (delta 0), reused 0 (delta 0)
			To https://github.com/Jakob37/GitCourse.git
			* [new tag]         v1.0 -> v1.0
			# The following command would push all tags at once
			$ git push --tags
			... output
		\end{lstlisting}
	\end{codebox}
	
	Tagging of commits are described in further detail at the following page: \\
	
	\url{https://git-scm.com/book/en/v2/Git-Basics-Tagging} \\
	
	%Reasons on why pushing multiple tags using the \verb$--follow-tags$ is preferred to simply pushing all tags with the \verb$--tags$ flag is discussed on: \\
	%
	%http://stackoverflow.com/questions/5195859/push-a-tag-to-a-remote-repository-using-git
	
	% https://git-scm.com/book/en/v2/Git-Basics-Tagging
	
	\newpage
	\section{Exercises}
	
	\subsection{Ignoring files with .gitignore}
	
	\begin{enumerate}
		\item Setup a .gitignore file within your repository. Try adding a file to the file tree, and then specifically excluding it in the .gitignore file.
		\item Ignore all files with a particular file ending. Then, use the \verb$git rm --cached$ command to remove all already tracked files (that are included in the repository) with that particular file ending.
		\item Create a directory named "Data" within your file tree, and create one or more files within it. Setup your repository so that you freely can generate data files within the directory without they being tracked by the repository.
	\end{enumerate}
	
	\subsection{Tagging commits}
	
	\begin{enumerate}
		\item Create a tag for your current commit, and push it to the remote.
		\item Continue developing your repository by adding at least another commit.
		\item List all your current tags, and investigate your recently created tag using the \verb$git show$ command.
		\item Check out the state of your files at the tag, before going back to your main.
	\end{enumerate}
	
	\newpage
	\section{Further reading}
	
	You have now reached the end of this material. Hopefully, you now have an understanding
	of the core concepts of Git, and are comfortable enough with Git to be able to set up and run your own repositories.
	
	A good way to gain fluency in Git is to use Git. Start using source control for your
	software (or text documents), and host it on a service like GitHub or Bitbucket. When you encounter problems, you will be able to find the solution on the internet (or in this material). Hopefully you now have some understanding of what is going on, and will be well prepared to investigate and solve future issues.
	
	If you have some extra time, take a look into the further reading of this chapter (see below).
	The first article on 'Project structure' is recommended in particular.
	
	Good luck with your future versioning!
	
	\subsection{Project structure}
	
	If you are interested in how you can approach
	organizing your bioinformatic projects, you are warmly recommended to read the article: \textit{A Quick Guide To Organizing Computational Biology Projects}
	by William Stafford Noble. Here, different approaches to making the project structure more manageable are discussed. In particular, note the section
	\textit{The Value of Version Control}. \\
	
	\url{http://journals.plos.org/ploscompbiol/article/asset?id=10.1371\%2Fjournal.pcbi.1000424.PDF}
	
	\subsection{Merge-tools}
	
	If working in complex projects and you are merging an extensive new feature into your main, you can end up with complex merge conflicts. Merge-tools are software designed to help you manage those and set up the code in the way you want.
	
	One popular merge-tools is \verb$Meld$ which provides a nice graphical interface for the merging: \\
	
	\url{http://meldmerge.org/} \\
	
	If you want to stay purely terminal-based (and is comfortable using the \verb$vim$/\verb$vi$ editors), then \verb$vimdiff$ is an alternative. \\
	
	\url{http://www.rosipov.com/blog/use-vimdiff-as-git-mergetool/}
	
\end{document}
